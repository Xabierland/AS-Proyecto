\documentclass{report}
\usepackage[spanish]{babel}
\usepackage[utf8]{inputenc}
\usepackage{graphicx, longtable, float, titlesec, hyperref, enumitem}
\usepackage[left=4cm, right=4cm, top=3cm, bottom=3cm]{geometry}

\hypersetup{
    hidelinks = true
}

\titleformat{\chapter}[display]
  {\normalfont\bfseries}{}{0pt}{\Huge}

\begin{document}
    \begin{titlepage}
        \centering
        \includegraphics[width=0.6\textwidth]{./img/logo.jpg}\\
        \vspace{1cm}
        \LARGE Administración de Sistemas\\
        \vspace{0.5cm}
        \Large Ingeniería Informática de Gestión y Sistemas de Información\\
        \vspace{3cm}
        \Huge Coches.xyz\\
        \vspace{2.5cm}
        \Large Autor:\\
        \vspace{0.2cm}
        \large Xabier Gabiña Barañano\\
        \vfill
        \today
    \end{titlepage}

    \tableofcontents
    \chapter{Introducción}
        Esta tarea tiene como objetivo explorar la implementación de una aplicación web con una base de datos incorporada, así como su despliegue utilizando Docker y Kubernetes. Estos dos entornos tecnológicos ofrecen soluciones fundamentales para abordar los desafíos actuales en el desarrollo de aplicaciones web al proporcionar un conjunto de herramientas y prácticas que simplifican la orquestación, escalabilidad y seguridad de dichas aplicaciones.\\

        Docker, como plataforma de contenedores, permite a los desarrolladores encapsular aplicaciones y sus dependencias en entornos aislados. Esta metodología de encapsulamiento elimina problemas de compatibilidad, asegura la consistencia y facilita la migración de aplicaciones entre diferentes entornos. Además, Docker posibilita una gestión eficiente de recursos, contribuyendo así al despliegue efectivo de aplicaciones web.\\
        
        Por otro lado, Kubernetes emerge como la solución definitiva para la orquestación de contenedores en clústeres. Desarrollada por Google, esta plataforma simplifica la administración y escalabilidad de aplicaciones web a gran escala. Kubernetes automatiza la distribución de contenedores, garantizando su disponibilidad, escalabilidad y equilibrio de carga, aspectos esenciales para aplicaciones que deben manejar un alto volumen de tráfico o que requieren una alta disponibilidad.\\
        
        La combinación de Docker y Kubernetes se ha vuelto esencial para implementar aplicaciones web modernas, proporcionando una base sólida para el desarrollo, despliegue y gestión de sistemas en un entorno altamente dinámico y desafiante. Este enfoque promete optimizar los recursos, reducir los tiempos de despliegue y asegurar la confiabilidad y seguridad de las aplicaciones en un mundo cada vez más centrado en la nube.\\
        
        Este informe se sumergirá en la creación de una aplicación web con una base de datos y su posterior despliegue mediante Docker y Kubernetes.

        \vfill
        \begin{center}
            \href{https://github.com/Xabierland/AS-Proyecto}{GitHub del proyecto}
        \end{center}
    \chapter{Descripción de la aplicación}
        El proyecto consiste en un entorno web enfocado a la compra y venta de coches de segunda mano.
        Una vez accedas a la pagina web podras ver un listado de vehiculos en venta, con los datos del vehiculo y del vendedor para poder comunicarte con el.
        Ademas, podras crearte una cuenta e iniciar sesion para poder publicar tu mismo un vehiculo y ponerlo a la venta.\\

        Para almacenar tantos los usuarios como los anuncios que estos generan se ha implementado una base de datos PostgreSQL.\\

        Dado que esto es un proyecto y por lo tanto no existen otros usuarios, se ha creado un programa en Python que cada cinco minutos genera un anuncio en la web.
        Para ello el programa se comunica con una API de modelos de coches y obtiene un modelo aleatorio con el que creara un anuncio.\\

        Ademas, para facilitar el desarroyo y mantenimiento de la aplicacion se ha agregado un panel de administracion para la base de datos.
        En este caso se ha elegido Adminer, una herramienta web que permite gestionar diferentes sistemas de gestion de bases de datos entre los que se incluye PostgreSQL.\\

        Para acabar, se ha implementado un gestor de claves llamado Vault. Este gestor nos permite almacenar de forma segura las claves de nuestro proyecto sin la necesidad de tenerlas escritas explicitamente en el codigo fuente de nuestra aplicacion.
    \chapter{Listado de las tareas realizadas}
        \begin{itemize}
            \item Desarroyo de una aplicación web.
            \item Creacion de una imagen Docker propia.
            \item Creacion de un entorno Docker Compose.
            \item Creacion de un despliegue Kubernetes equivalente.
            \item Inclusion de imágenes adicionales.
            \item Utilizacion de funcionalidades Docker/Kubernetes no vistas en clase.
        \end{itemize}
    \chapter{Explicaciones de las tareas realizadas}
        \section{Desarroyo de una aplicación web.}
            El desarroyo de la aplicacion web se puede dividir en dos partes, el backend y el frontend.\\

            El frontend se ha desarroyado utilizando HTML y CSS.
            Para facilitar el desarroyo se ha utilizado Bootstrap, que nos permite crear paginas web responsive de forma sencilla.\\

            El backend se ha desarroyado utilizando PHP.
            PHP se utiliza para gestionar las conexiones de la base de datos como puede ser el registrar o inicio de sesion de un usuario, la creacion de anuncios y la obtencion y muestra de los datos de los anuncios.\\

            Esta aplicacion se ha implementado en un servidor web Apache2 en un contenedor Docker. Para ello lo primero ha sido usar la imagen de php llamada php:8.2-apache la cual ya incluye Apache2 y PHP. 
            Aun asi, ha sido necesario instalar de forma algunas librerias como el pdo de PostgreSQL para que PHP se pudiese conectar despues con la base de datos. 
            Despues solamente ha sido necesario copiar los ficheros de la aplicacion en el directorio /var/www/html/ del contenedor con los permisos correctos y activar un modulo de apache2.\\
        \clearpage
        \section{Creacion de una imagen Docker propia.} 
        \clearpage
        \section{Creacion de un entorno Docker Compose.}
        \clearpage
        \section{Creacion de un despliegue Kubernetes equivalente.}
        \clearpage
        \section{Inclusion de imágenes adicionales.}
        \clearpage
        \section{Utilizacion de funcionalidades Docker/Kubernetes no vistas en clase.}
    \chapter{Declaración sobre uso de asistentes virtuales}
        Para la realizacion de este proyecto se ha utilizado dos asistentes virtuales, GPT-3.5 y GitHub Copilot.
        Ambos se han usado principalmente para la tarea de picado de codigo resolucion de errores.
    \chapter{Bibliografia}
        \begin{itemize}
            \item GPT-3.5. (2023). Respuesta a una pregunta sobre PHP. OpenAI. \url{https://www.openai.com/}
            \item GitHub Copilot. (2022). Autocompletado. GitHub. \url{https://github.com/features/copilot}
            \item Docker. (2021). Documentación. Docker. \url{https://docs.docker.com/}
            \item Kubernetes. (2021). Documentación. Kubernetes. \url{https://kubernetes.io/docs/home/}
            \item PHP. (2021). Documentación. PHP. \url{https://www.php.net/docs.php}
            \item PostgreSQL. (2021). Documentación. PostgreSQL. \url{https://www.postgresql.org/docs/}
            \item Apache. (2021). Documentación. Apache. \url{https://httpd.apache.org/docs/}
            \item Minikube (2023). Documentación Minikube. \url{https://minikube.sigs.k8s.io/docs/handbook/}
        \end{itemize}
\end{document}