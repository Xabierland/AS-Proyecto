\documentclass{report}
\usepackage[spanish]{babel}
\usepackage[utf8]{inputenc}
\usepackage{graphicx, longtable, float, titlesec, hyperref, enumitem, dingbat}
\usepackage[dvipsnames]{xcolor}
\usepackage[margin=4.3cm]{geometry}

\hypersetup{
    hidelinks = true
}

\titleformat{\chapter}[display]
  {\normalfont\bfseries}{}{0pt}{\Huge}

\begin{document}
    \begin{titlepage}
        \centering
        \includegraphics[width=0.6\textwidth]{./img/logo.jpg}\\
        \vspace{1cm}
        \LARGE Administración de Sistemas\\
        \vspace{0.5cm}
        \Large Ingeniería Informática de Gestión y Sistemas de Información\\
        \vspace{3cm}
        \Huge Coches.xyz\\
        \vspace{2.5cm}
        \Large Autor:\\
        \vspace{0.2cm}
        \large Xabier Gabiña Barañano\\
    \end{titlepage}
    \tableofcontents
    \chapter{Introducción}
        Esta tarea tiene como objetivo explorar la implementación de una aplicación web con una base de datos incorporada, así como su despliegue utilizando Docker y Kubernetes. Estos dos entornos tecnológicos ofrecen soluciones fundamentales para abordar los desafíos actuales en el desarrollo de aplicaciones web al proporcionar un conjunto de herramientas y prácticas que simplifican la orquestación, escalabilidad y seguridad de dichas aplicaciones.\\

        Docker, como plataforma de contenedores, permite a los desarrolladores encapsular aplicaciones y sus dependencias en entornos aislados. Esta metodología de encapsulamiento elimina problemas de compatibilidad, asegura la consistencia y facilita la migración de aplicaciones entre diferentes entornos. Además, Docker posibilita una gestión eficiente de recursos, contribuyendo así al despliegue efectivo de aplicaciones web.\\
        
        Por otro lado, Kubernetes emerge como la solución definitiva para la orquestación de contenedores en clústeres. Desarrollada por Google, esta plataforma simplifica la administración y escalabilidad de aplicaciones web a gran escala. Kubernetes automatiza la distribución de contenedores, garantizando su disponibilidad, escalabilidad y equilibrio de carga, aspectos esenciales para aplicaciones que deben manejar un alto volumen de tráfico o que requieren una alta disponibilidad.\\
        
        La combinación de Docker y Kubernetes se ha vuelto esencial para implementar aplicaciones web modernas, proporcionando una base sólida para el desarrollo, despliegue y gestión de sistemas en un entorno altamente dinámico y desafiante. Este enfoque promete optimizar los recursos, reducir los tiempos de despliegue y asegurar la confiabilidad y seguridad de las aplicaciones en un mundo cada vez más centrado en la nube.\\
        
        Es posible encontrar comandos y claves de acceso para la utilización de la aplicación en el README del repositorio de GitHub del proyecto.\\
        \vfill
        \begin{center}
            \textcolor{Cyan}{\href{https://github.com/Xabierland/AS-Proyecto}{GitHub del proyecto}}
        \end{center}
    \chapter{Descripción de la aplicación}
        El proyecto consiste en un entorno web enfocado a la compra y venta de coches de segunda mano.
        Una vez accedas a la página web podrás ver un listado de vehículos en venta, con los datos del vehículo y del vendedor para poder comunicarte con él.
        Además, podrás crearte una cuenta e iniciar sesión para poder publicar tú mismo un vehículo y ponerlo a la venta.\\

        Para almacenar tantos los usuarios como los anuncios que estos generan se ha implementado una base de datos PostgreSQL.\\

        Dado que esto es un proyecto y por lo tanto no existen otros usuarios, se ha creado un programa en Python que cada minuto genera un anuncio en la web.
        Para ello el programa se comunica con una API de modelos de coches y obtiene un modelo aleatorio con el que creara un anuncio.\\

        Como añadido, para facilitar el desarrollo y mantenimiento de la aplicación se ha agregado un panel de administración para la base de datos.
        En este caso se ha elegido Adminer, una herramienta web que permite gestionar diferentes sistemas de gestión de bases de datos entre los que se incluye PostgreSQL.\\

        Para acabar, se ha implementado un gestor de claves llamado Vault. Este gestor nos permite almacenar de forma segura las claves de nuestro proyecto sin la necesidad de tenerlas escritas explícitamente en el código fuente de nuestra aplicación.\\
    \chapter{Listado de las tareas realizadas}
        \begin{itemize}
            \item Tareas obligatorias:
            \begin{itemize}
                \item Desarrollo de una aplicación web. \checkmark
                \item Creación de archivos Dockerfile y otras imágenes. \checkmark
                \item Creación de un entorno Docker Compose. \checkmark
            \end{itemize}
            \item Tareas opcionales:
            \begin{itemize}
                \item Creación de un despliegue Kubernetes equivalente. \checkmark
                \item Inclusión de imágenes adicionales. \checkmark
                \item Utilización de funcionalidades Docker o Kubernetes no vistas en clase. \checkmark
            \end{itemize}
        \end{itemize}
    \chapter{Explicaciones de las tareas realizadas}
        \section{Desarrollo de una aplicación web.}
            El desarrollo de la aplicación web se puede dividir en tres partes, el backend, el frontend y la base de datos.
            El frontend se ha desarrollado utilizando HTML y CSS.
            Para facilitar el desarrollo se ha utilizado Bootstrap, que nos permite crear páginas web responsive de forma sencilla.
            El backend se ha desarrollado utilizando PHP.
            PHP se utiliza para gestionar las conexiones de la base de datos como puede ser el registrar o inicio de sesión de un usuario, la creación de anuncios y la obtención y muestra de los datos de los anuncios.\\
            
            Para almacenar la información de la web he usado PostgreSQL.
            PostgreSQL es un sistema de gestión de bases de datos relacional orientado a objetos y de código abierto.\\

            Dado que la página web es un proyecto y por lo tanto no existen otros usuarios, he decidido crear un programa que cada minuto genera un anuncio en la web.
            De esta forma se generan anuncios de forma automática y se simula el funcionamiento de la web.
        \clearpage
        \section{Creación de archivos Dockerfile y otras imagenes.}
            Una vez creada la aplicación web, la base de datos y el bot, he creado los archivos Dockerfile para crear las imágenes de Docker.
            He creado un archivo Dockerfile para cada imagen dado que todo necesita archivos o variables para funcionar correctamente y pensando en el posterior despliegue Kubernete.\\

            Para la creación del frontend y backend de la web he optado por la imagen oficial de PHP con Apache.
            Una vez hecho esto he instalado las dependencias de PHP para la comunicación con la base de datos, he copiado los ficheros de la web en el directorio \texttt{/var/www/html} y he expuesto el puerto 80.
            He otorgado los permisos www-data al directorio.
            \begin{center}
                \textcolor{Cyan}{\href{https://hub.docker.com/repository/docker/xabierland/web/general}{Web}}
            \end{center}

            Para la creación de la base de datos he optado por la imagen oficial de PostgreSQL.
            Aquí únicamente he definido las variables de entorno necesarias para la creación de la base de datos y he expuesto copiado el fichero de la base de datos en el directorio \texttt{/docker-entrypoint-initdb.d}.
            \begin{center}
                \textcolor{Cyan}{\href{https://hub.docker.com/repository/docker/xabierland/sgbd/general}{SGBD}}
            \end{center}
            Por último, para la creación del bot he optado por la imagen oficial de Python Slim.
            He optado por esta imagen debido a que es una imagen muy ligera.
            Una vez hecho esto he instalado las dependencias necesarias para la comunicación con la API de Vault y la API de \href{https://carapi.app/api#/}{carapi}.
            \begin{center}
                \textcolor{Cyan}{\href{https://hub.docker.com/repository/docker/xabierland/robot/general}{Robot}}
            \end{center}
            Todavía quedan dos imágenes por nombrar que serán vistas más adelante en la sección 4.5.\\
        \clearpage
        \section{Creación de un entorno Docker Compose.}
            Para orquestar el levantamiento de los contenedores a partir de las imágenes ya mencionadas y las que no se han mencionado aun he utilizado Docker Compose.
            Para ello he creado un archivo docker-compose.yaml en el que he definido los servicios que quiero levantar.\\

            En el caso de la web he definido un servicio llamado web que inicia el contenedor con la imagen del frontend y el backend a partir del Dockerfile ya mencionado.
            Además, he definido un volumen de host junto con ALLOW\_OVERRIDE para que los cambios en el código se vean reflejados en el contenedor a tiempo real.
            El puerto por defecto de la web es el 80 y es el que he utilizado en el docker-compose.yaml.\\
            
            En el caso de la base de datos he definido un servicio llamado SGBD que inicia el contenedor con la imagen de PostgreSQL a partir del Dockerfile ya mencionado.
            En este caso, he definido un volumen llamado SGBD-data para que los datos de la base de datos no se pierdan al reiniciar el contenedor.
            El puerto por defecto de la base de datos es el 5432 y es el que he utilizado en el docker-compose.yaml.\\
            
            En el caso del bot he definido un servicio llamado robot que inicia el contenedor con la imagen de Python a partir del Dockerfile ya mencionado.
            No ha sido necesario definir nada más dado que nadie se conecta al bot ni necesita almacenar datos.\\

            Al igual que en la sección anterior, todavía quedan dos servicios por nombrar que serán vistas más adelante en la sección 4.5.\\
        \clearpage
        \section{Creación de un despliegue Kubernetes equivalente.}
            \textbf{Aviso:} Para la realización de esta tarea he utilizado Minikube. Minikube es una herramienta que nos permite crear un clúster de Kubernetes en local. El formato del .yaml del Ingress no es exactamente igual en Kubernetes de GCP que el de Minikube, ten esto en cuenta si vas a desplegarlo en GCP. También tener en cuenta que Minikube no trae los Ingress activados por defectos y debe ser activados mediante:
            \begin{center}
                \texttt{minikube addons enable ingress}
            \end{center}
            Para el despliegue he usado cinco objetos de Kubernetes, Deployment, Service, PersistentVolume, PersistentVolumeClaim e Ingress.\\

            El Deployment es el objeto que se encarga de crear los pods y de asegurarse de que estos estén en el estado y numero deseado.
            He creado uno por cada imagen que he creado asignándole un nombre, una imagen de mi DockerHub y el puerto en caso de que haga uso de uno.\\

            El Service es el objeto que se encarga de asignar una IP y un puerto a los pods para que puedan ser identificados.
            He creado uno para cada imagen excepto para el robot ya que este no necesita ser identificado.
            Estos servicios son de tipo ClusterIP, lo que significa que solo pueden ser accedidos desde dentro del clúster.\\

            El PersistentVolume es el objeto que se encarga de crear un volumen persistente en el nodo para que los datos no se pierdan al reiniciar el pod.
            He creado uno para la base de datos y otro para las fotos de la web asignándole un nombre, un tamaño y un tipo de almacenamiento.
            El tipo de almacenamiento es hostPath, lo que significa que el volumen se crea en el nodo y no en un servicio externo.\\

            El PersistentVolumeClaim es el objeto que se encarga de reclamar el volumen persistente para que pueda ser utilizado por el pod.\\

            El Ingress es el objeto que se encarga de enrutar el tráfico a los servicios correspondientes.
            He creado uno con redirecciones a la web y otro al panel de administración de la base de datos.
            Para ello he definido las reglas de enrutamiento y el servicio al que deben ser redirigidas.
            En caso de no especificar ruta se redirige a la web mientras que usando /db se redirige al panel de administración de la base de datos.\\
        \clearpage
        \section{Inclusión de imágenes adicionales.}
            Como he mencionado previamente, las imágenes adicionales que he utilizado son Adminer y Vault.\\
            
            Adminer es una herramienta web que nos permite gestionar diferentes sistemas de gestión de bases de datos entre los que se incluye PostgreSQL.
            Para la creación de la imagen he utilizado la imagen oficial de Adminer, sin necesidad de configuración o un Dockerfile.
            Esta imagen se ha utilizado en un servicio que se ha añadido al docker-compose.yaml exponiendo el puerto 8080.
            \begin{center}
                \textcolor{Cyan}{\href{https://hub.docker.com/repository/docker/xabierland/sabd/general}{SABD}}
            \end{center}
            Vault es un gestor de claves que nos permite almacenar de forma segura las claves de nuestro proyecto sin la necesidad de tenerlas escritas explícitamente en el código fuente de nuestra aplicación.
            En este caso si he usado un archivo Dockerfile.
            Como base he utilizado la imagen oficial de Vault.
            Para la configuración de Vault he creado un archivo .hcl en el que se definen la configuración de inicio de Vault y lo he copiado dentro de la imagen. 
            Esto es necesario dado que Vault, por defecto, almacena las claves en memoria y por lo tanto se pierden al reiniciar el contenedor.
            Entre las configuraciones que he definido se encuentra el almacenamiento de claves en ficheros y la configuración de red de la API.
            También he creado un archivo .sh que se encarga de iniciar Vault y de desbloquear el almacén de claves.
            Una vez generada la imagen se ha añadido un servicio al docker-compose.yaml exponiendo el puerto 8200.
            \begin{center}
                \textcolor{Cyan}{\href{https://hub.docker.com/repository/docker/xabierland/vault/general}{Vault}}
            \end{center}
        \clearpage
        \section{Utilización de funcionalidades Docker o Kubernetes no vistas en clase.}
            Como funcionalidad no vista en clase he utilizado el networking de Docker.
            Definir el networking de los contenedores es una parte muy importante y útil de Docker.
            La principal funcionalidad que nos ofrece es la de aislar los contenedores de forma que no puedan comunicarse más que con aquellos contenedores que nosotros queramos.
            Esto puede ser muy útil si tenemos una API para comunicar la base de datos con la web ya que de esta forma la base de datos no podría ser accesible desde la web lo que impediría que un atacante pudiera acceder a ella.
            Por desgracia, mi aplicación web tiene el backend y el frontend en el mismo contenedor por lo que no he podido aprovechar esta funcionalidad al 100\%.
            Aun así, he añadido la funcionalidad de networking como prueba de concepto.
            La configuración de la red de Docker puede ser vista en el archivo docker-compose.yaml.\\

            Además, he utilizado también la funcionalidad HEALTHCHECK.
            Esta función nos permite comprobar el estado de un contenedor y frenar el arranque de otros que dependan del mismo para evitar posibles errores.
            He añadido esta funcionalidad al Dockerfile de la base de datos para comprobar que la base de datos está funcionando correctamente antes de que el robot, Adminer y la web intenten conectarse a ella.
            La configuración de HEALTHCHECK puede ser vista en el archivo Dockerfile de la base de datos.
            Además, se ha añadido un \texttt{depends\_on} en el docker-compose.yaml para que los contenedores no se inicien hasta que la base de datos esté funcionando correctamente.\\
        \clearpage
    \chapter{Declaración sobre uso de asistentes virtuales}
        Para la realización de este proyecto se ha utilizado dos asistentes virtuales, GPT-3.5 y GitHub Copilot.\\

        El uso que le he dado a GitHub Copilot ha sido bastante limitado.
        La única tarea que ha desempeñado ha sido la de autocompletado de código que aporta mediante su integración con Visual Studio Code.\\

        Por otro lado, el uso que le he dado a GPT-3.5 ha sido bastante más amplio.
        Su uso principal ha sido el de ayudarme en las tareas de debugging mediante la interpretación de log y mensajes de error.
        También lo he utilizado para preguntas más generales y para aportarme ideas sobre que podía implementar en el proyecto.
        Por último, la corrección de errores ortográficos y gramaticales en este informe.\\
    \chapter{Bibliografia}
        \begin{itemize}
            \item GPT-3.5. (2023). Respuesta a una pregunta sobre PHP. OpenAI. \url{https://www.openai.com/}
            \item GitHub Copilot. (2022). Autocompletado. GitHub. \url{https://github.com/features/copilot}
            \item Docker. (2021). Documentación. Docker. \url{https://docs.docker.com/}
            \item Kubernetes. (2021). Documentación. Kubernetes. \url{https://kubernetes.io/docs/home/}
            \item Minikube (2023). Documentación Minikube. \url{https://minikube.sigs.k8s.io/docs/handbook/}
            \item Apache. (2021). Documentación. Apache. \url{https://httpd.apache.org/docs/}
            \item PHP. (2021). Documentación. PHP. \url{https://www.php.net/docs.php}
            \item PostgreSQL. (2021). Documentación. PostgreSQL. \url{https://www.postgresql.org/docs/}
            \item Vault (2023). Documentación. Vault. \url{https://developer.hashicorp.com/vault/docs} 
        \end{itemize}
\end{document}