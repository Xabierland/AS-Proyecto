\documentclass{report}
\usepackage[spanish]{babel}
\usepackage[utf8]{inputenc}
\usepackage{graphicx, longtable, float, titlesec, hyperref}

\hypersetup{
    hidelinks = true
}

\titleformat{\chapter}[display]
  {\normalfont\bfseries}{}{0pt}{\Huge}

\begin{document}
    \begin{titlepage}
        \centering
        \includegraphics[width=0.6\textwidth]{./img/logo.jpg}\\
        \vspace{1cm}
        \LARGE Administración de Sistemas\\
        \vspace{0.5cm}
        \Large Ingeniería Informática de Gestión y Sistemas de Información\\
        \vspace{3cm}
        \Huge Coches.xyz\\
        \vspace{2.5cm}
        \Large Autor:\\
        \vspace{0.2cm}
        \large Xabier Gabiña Barañano\\
        \vfill
        \today
    \end{titlepage}

    \tableofcontents
    \chapter{Introducción}
        La presente tarea pretende explorar la implementación de una aplicación web con una base de datos incorporada, y su despliegue utilizando Docker y Kubernetes. Estos dos entornos tecnológicos ofrecen soluciones esenciales para desafíos contemporáneos en el desarrollo de aplicaciones web, al proporcionar un conjunto de herramientas y prácticas que simplifican la orquestación, escalabilidad y seguridad de dichas aplicaciones.\\

        Docker, una plataforma de contenedores, permite a los desarrolladores encapsular aplicaciones y sus dependencias en entornos aislados. Esta metodología de encapsulamiento elimina problemas de compatibilidad, garantiza la consistencia y facilita la migración de aplicaciones entre distintos entornos. Además, Docker permite una gestión eficiente de recursos, lo que contribuye al despliegue eficaz de aplicaciones web.\\
        
        Kubernetes, por su parte, se alza como la solución definitiva para la orquestación de contenedores en clústeres. Esta plataforma, desarrollada por Google, simplifica la administración y escalabilidad de aplicaciones web a gran escala. Kubernetes automatiza la distribución de contenedores, garantizando su disponibilidad, escalabilidad y equilibrio de carga, lo que resulta esencial para aplicaciones que deben manejar un alto volumen de tráfico o que requieren una alta disponibilidad.\\
        
        La combinación de Docker y Kubernetes se ha vuelto esencial para implementar aplicaciones web modernas, ya que proporciona una base sólida para desarrollar, desplegar y gestionar sistemas en un entorno altamente dinámico y demandante. Este enfoque promete optimizar los recursos, reducir los tiempos de despliegue y garantizar la confiabilidad y la seguridad de las aplicaciones en un mundo cada vez más orientado a la nube.\\
        
        El presente informe se sumergirá en la creación de una aplicación web con una base de datos y el posterior despliegue mediante Docker y Kubernetes.
    \chapter{Descripción de la aplicación}
    \chapter{Tareas realizadas}
        \section{Desarrollar una aplicación web}
            La primera tarea a realizar en este proyecto es la realizacion de una aplicacion web.
            En mi caso, como ya he comentado en la descripción de la aplicación, he decidido realizar una aplicación web para la compra y venta de coches de segunda mano.
            Dicha aplicacion web trabajara con las siguientes tecnológiás:
            \begin{itemize}
                \item HTML
                \begin{itemize}
                    \item Usado para generar la estructura de la pagina Web.
                \end{itemize}
                \item CSS
                \begin{itemize}
                    \item Usado para dar estilo a la pagina Web.
                    \item Se ha utilizado la plantilla de Bootstrap.
                \end{itemize}
                \item PostgreSQL
                \begin{itemize}
                    \item Usado para la base de datos.
                    \item Almacena los datos de los usuarios y los coches.
                    \item Se ha instalado un gestor de base de datos Adminer.
                \end{itemize}
                \item Apache
                \begin{itemize}
                    \item Usado para el servidor web.
                    \item Se le ha instalado PHP y varias extensiones para comunicarse con la base de datos.
                \end{itemize}
                \item PHP
                \begin{itemize}
                    \item Se utiliza tanto para obtener datos como para enviarlos al servidor de base de datos.
                    \item Tambien se encarga de la seguridad de la pagina saneando los campos y hasheando las entradas.
                \end{itemize}
            \end{itemize}
            El codigo de esta aplicacion se puede encontrar en \url{https://github.com/Xabierland/AS-Proyecto/tree/master/src}
        \clearpage
        \section{Implementación de imágenes Docker}
            
        \clearpage
        \section{Creacion de entorno Docker Compose}
        \clearpage
        \section{Creacion de despliegue Kubernetes}
        \clearpage
        \section{Inclusion de images extras}
        \clearpage
        \section{Nuevas funcionalidades}
    \chapter{Declaración sobre asistentes virtuales}
        Para la realización de este proyecto se han utilizado principalemente dos asistentes virtuales, ChatGPT de OpenAI y GitHub Copilot de GitHub.\\

        El uso de ChatGPT ha sido principalmente para tareas de picado de codigo repititivas y para obtener ideas para implementar en la aplicación.\\

        El uso de GitHub Copilot ha sido en cambio el autocompletado que ofrece desde Visual Studio Code para agilizar el desarrollo y para facilitar en gran parte la escritura en ficheros .yaml.\\
    \chapter{Bibliografia}
    \begin{itemize}
        \item GPT-3.5. (2023). Respuesta a una pregunta sobre PHP. OpenAI. \url{https://www.openai.com/}
        \item GitHub Copilot. (2022). Autocompletado. GitHub. \url{https://github.com/features/copilot}
        \item Docker. (2021). Documentación. Docker. \url{https://docs.docker.com/}
        \item Kubernetes. (2021). Documentación. Kubernetes. \url{https://kubernetes.io/docs/home/}
        \item PHP. (2021). Documentación. PHP. \url{https://www.php.net/docs.php}
        \item PostgreSQL. (2021). Documentación. PostgreSQL. \url{https://www.postgresql.org/docs/}
        \item Apache. (2021). Documentación. Apache. \url{https://httpd.apache.org/docs/}
        \item Minikube (2023). Documentación Minikube. \url{https://minikube.sigs.k8s.io/docs/handbook/}
    \end{itemize}
\end{document}